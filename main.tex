\documentclass[a4j,10pt]{jsarticle}
    \usepackage{layout,url,resume}
    \usepackage[dvipdfmx]{graphicx}
    \pagestyle{plain}

    \begin{document}
    %\layout

    \title{
        WIP 2018年春学期 \\
        RDMAによるLinuxのコンテキスト復元手法
    }

    \author{
        Arch B3 石川 達敬 (tatsu)\thanks{慶應義塾大学 環境情報学部}
        \and
        Adviser: 空閑 洋平 (sora)\thanks{慶應義塾大学大学院 政策・メディア研究科特任講師}
    }

    \maketitle

    \section{概要}

    PCIe経由で取得した物理アドレス空間を解析し,リモートホストのOSのコンテキストと実行中のプロセスを復元するために必要な情報の整理と,手法の提案.

    \section{背景と目的}

    OSデバッグやセキュリティフォレンジックでは,物理・仮想メモリを解析していく際に,仮想マシンをたて,実行環境をサンドボックス化し,デバッグしていく手法が一般的であった.
    デバッグの際に用いられる環境及びツールとして,Xen,あるいはQEMU+KVMがあげられる.また,関連ソフトウェアとして,libvmi\cite{libvmi}やgoogle/rekallなどもあげられる.
    この手法の課題としてあげられるのが,メモリ解析用のプロセスが常駐し,解析対象のオーバヘッドになる,という点だ.

    この課題を解決するために,本研究では,外部から物理メモリの解析手法を確立することで,対象ホストの物理・仮想メモリの監視アーキテクチャを構築し,ゼロ・オーバヘッドのメモリ監視環境の実現を目指す.

    \section{本研究のアプローチ}

    そこで,本研究のアプローチとして,RDMAを用いて,外部から対象ホストの物理メモリ空間を調査・解析し,ホスト上で動作するLinuxの仮想メモリを復元,プロセスを監視するという手法を選んだ.
    MS-DOS及び,Linuxの簡単なプログラムにて,動作することを確認した後に,LinuxのMMUの機構を調査し,MMUを逆算して行く.
    % この手法は,最初に,UNIX V6で実行可能かどうかを検証する.一度UNIX V6で実験する理由は,仮想アドレスの変換方法がLinuxに比べ単純,という点があげられる.UNIX V6で動作することを確認し検証したのち,Linux版を実装する.

    本研究に用いるツールとしてsoraさんが実装した,PCIe-Eth bridgeを使用する.PCIe-Eth bridgeは,EthernetフレームをPCIeメッセージに変換するFPGAデバイスであり,これを使用することで,ローカルホストからEthernetフレームをソフトウェアで送信することで,リモートホスト上のPCIeメッセージに変換され,リモートホストの物理メモリのデータを読み書きすることができるようになる.

    \subsection{本研究のメリット}

    本研究のメリットは二点ある.
    一点目は,PCIeバスを利用するため,リモートホストのCPUリソースを使わないという点.
    二点目は,リモートホストがカーネルパニックなどを起こした際にも,解析が可能となるという点である.

    \subsection{本研究の課題}

    本研究手法では,物理メモリ空間にアクセス可能となる.
    しかし,解析対象の実際に動いているOSでは,MMUを用いた仮想メモリ空間上で各プロセスは動作している.
    したがって,物理メモリ空間を探索することで外部から仮想メモリの状態を把握しなければならない.
    その際に,ブートローダをはじめとして,OSのコンテキストを追う必要がある.
    仮想メモリ空間上にあるメモリが,実際にどの物理アドレスに配置されているのかを逆算し,各プロセスの状態をローカルホストで復元していく.\cite{unixv6}

    Linuxの場合,MMUの方式がUNIX V6に比べ,複雑なため,本研究では,MMUの方式のうち,セグメント機構に焦点を当てる.

    \section{手法}

    本研究ではOSのコンテキスト,プロセスの状態を復元することが目的であるため,プロセスを動かすために必要な情報を一見ランダムに見える物理アドレス空間から読み解いていく必要がある.
    Linuxにおけるセグメント回路の方式を逆算し,論理アドレスを求める.

    \section{評価}

    リモートホストのパフォーマンスを落とすことなく監視ができることを実証.\cite{linuxkernel}

    \section{参考文献}

    \begin{thebibliography}{99}
        \bibitem{libvmi}
		libvmi.com
        \texttt{http://libvmi.com/}

        \bibitem{unixv6}
        青柳 隆宏(2013) . はじめてのOSコードリーディング ~UNIX V6で学ぶカーネルのしくみ

        \bibitem{linuxkernel}
        Daniel P. Bovet(2007) . 詳解 Linuxカーネル 第3版
    \end{thebibliography}

\end{document}
